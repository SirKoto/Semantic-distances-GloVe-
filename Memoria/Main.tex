\documentclass[catalan,10pt,a4paper]{article}
\usepackage[utf8]{inputenc}
\usepackage[T1]{fontenc}
\usepackage[catalan]{babel}
\usepackage{amsmath}
\usepackage{amsfonts}
\usepackage{amssymb}
\usepackage{graphicx}
\usepackage{mathtools}
\usepackage{listings}
\lstset{language=C++}
\usepackage{fancyhdr}
\usepackage[a4paper, total={6in, 8in}]{geometry}

\pagestyle{fancy}
\usepackage{tikz}
\usepackage[hidelinks,colorlinks=true, linkcolor=blue,citecolor={blue},urlcolor=blue]{hyperref}
\lhead{Projecte TGA: \textit{word2vec}}
\rhead{Antoni Casas Muñoz\\Pol Martín Garcia}

\begin{document}
	\begin{titlepage}
		\centering
		{\bfseries\LARGE Universitat Politècnica de Catalunya \par}
		\vspace{1cm}
		{\scshape\Large Facultat d'Informàtica de Barcelona\par}
		\vspace{3cm}
		{\scshape\Huge Distàncies \textit{word2vec} \par}
		\vspace{3cm}
		{\itshape\Large Projecte Targetes Gràfiques (TGA) \par}
		\vfill
		{\Large Antoni Casas Muñoz \par}
		{\Large Pol Martín Garcia \par}
		\vfill
		{\Large Maig 2020 \par}
	\end{titlepage}
	
	\newpage
	
\section*{Problema a resoldre}

El problema a solucionar es la computació de similitud de paraules utilitzant el model word2vec de GloVe \cite{GloVeGlo18:online} amb la mètrica de similitud de cosinus \cite{Cosinesi72:online}.

Word2Vec es una representació densa d'una paraula en un espai vectorial reduït, on la representació manté les analogies semàntiques amb operacions aritmètiques simples, com, per exemple, $\text{water} + \text{freeze} \simeq \text{ice}$, o $\text{king} - \text{man} + \text{woman} \simeq \text{queen}$. Això permet operar amb paraules, específicament amb els seus significats, igual que es podria operar amb altres tipus de dades permetent l'ús d'aquestes representacions per a múltiples tasques.

Mentre que per a convertir una paraula a la seva representació densa es tan simple com accedir a un diccionari i extreure el valor, sent un procés només intensiu en espai, presenten una complicació al fer la conversió de la representació densa (word2vec) a la representació esparsa (el vocabulari de l'idioma en què word2vec va ser entrenat) aquesta conversió és especialment difícil si s'han fet operacions aritmètiques amb aquesta representació. La manera de fer aquesta conversió és trobar la paraula, o paraules, més properes, i per això s'utilitza la similitud de cosí. No s'utilitza la distància euclidiana típica, ja que aquesta mètrica ofereix poc significat en espais amb un gran nombre de dimensions, com és el cas de molts models word2vec, en el nostre cas l'espai és de 300 dimensions i per tant no seria una mètrica vàlida.

La similitud de cosinus \cite{Cosinesi72:online} és una mètrica de similitud que és utilitzada per ser fàcil de computar, i oferir valors en el rang de [-1,1], sent 1 el valor que indica absoluta similitud Aquesta mètrica es computa com amb l'equació \ref{eq:cosSim}. La mètrica no mesura la distància en l'espai, sinó que mesura com de paral·leles són les representacions.
\begin{equation} \label{eq:cosSim}
	\text{cosSim}(\vec A,\vec B) = \frac{\vec A\times \vec B}{|\vec A|\cdot|\vec B|}
\end{equation}

\subsubsection*{Canvis al model}
Per a la implementació de l'algoritme, primer hem obtingut el model de word2vec de GloVe \cite{GloVeGlo18:online}, i l'hem modificat per ordenar alfabèticament les paraules i afegir les normes de cada representació densa al model propi, d'aquesta manera no és necessari computar la norma en cada operació de similitud. També, s'ha fet una millora en implementacions consecutives, on el model són dos fitxers, un amb les paraules ordenades, i un altre fitxer que és un binari amb les representacions denses. El model llavors està format per unes $2.2\cdot 10^{6}$ paraules, on cada paraula és una línia al fitxer, primer la paraula, després la norma, després els 300 valors de la representació densa.	

\section*{Implementació}
Hi ha dues versions correctament implementades del projecte, amb resultats finals equivalents.

Aquestes versions també són equivalents al codi seqüencial, trobat al fitxer \textit{main.cpp} en la funció \verb|sequentialSearch()|. Aquesta, donat un vector d'\textit{embeddings}, l'índex d'un d'aquests, i un nombre N, troba d'entre tots els \textit{embeddings} del vector els N més semblants a l'\textit{embedding} identificat per l'índex.

\subsection*{Algoritme}
Qualsevol dels algoritmes implementats per a solucionar aquest problema es basen en 3 parts.
\begin{enumerate}
	\item Trobar la paraula en el vector d'\textit{embeddings}. Aquest pas sempre es du a terme amb una cerca binària en CPU, per tant no el discutirem en aquest document.
	\item Dur a terme el còmput de les distàncies de cosinus (o similituds de cosinus) entre tots els \textit{embeddings} i l'\textit{embedding} de la paraula cercada.
	\item Filtrar els resultats, i escollir les N paraules més semblants (amb major similitud) a la paraula cercada amb les dades calculades, de manera ordenada.
\end{enumerate}

Els punts 2 i 3, es troben tan implementats per a CPU, a \textit{main.cpp}, com per GPU, a \textit{kernel.cu}.

\subsection*{Càlcul de similituds}
El càlcul de les distàncies o similituds es du a terme a GPU pel kernel \verb|DotProduct()|, el qual calcula el producte escalar amb cadascun dels \textit{embeddings}, i posteriorment en divideix el resultat pel producte de normes.

Això es du a terme movent el \textit{embeddings} de la paraula cercada a memòria \textit{shared}, i posteriorment l'usen tots els \textit{threads} del bloc per a dur a terme el producte escalar amb un altre \textit{embedding}.

D'aquesta manera cada \textit{thread} computa una sola similitud.

\subsection*{Filtrat i ordenació}
El filtrat per GPU requereix conèixer les N paraules amb una major similitud. Per a això s'ha dividit el còmput d'aquest procés en dues funcions.

La primera, \verb|FirstMerge()|, divideix el vector de similituds resultants en trossos de N elements, els quals són ordenats usant ordenació per inserció donat que N sempre serà un nombre petit. L'ordenació es du a terme \textit{on-place}, de manera que es reutilitza la mateixa memòria per emmagatzemar el resultat.\newline
D'aquesta forma, obtenim el vector de similituds en trossos de N elements internament ordenats. Evidentment, a part d'aquest vector de similituds s'emmagatzema un vector de índex a les paraules originals, per no perdre la relació entre valor de similitud i la respectiva paraula.

Seguida i finalment la funció \verb|BotchedMergeSort()| aprofita els segments ordenats per a dur a terme l'ordenació en una reducció del problema. Cada \textit{thread} compara dos dels pedaços de N elements preordenats en un de sol.\newline
Aquesta funció redueix el nombre de similituds a comparar a la meitat per cada crida, i es va usant fins que només resta un sol vector de N elements, el qual identifica les N paraules amb major similitud.

\subsection*{Canvis de versió}
Les millores en la segona versió del programa són separables en canvis en el codi del kernel, i en carrega de les dades a memòria.

Pel que fa al kernel, s'ha reduït l'espai de memòria reservat, utilitzant memòria local de cada \textit{thread} per emmagatzemar l'ordenació temporal en el mètode \verb|BotchedMergeSort()|. A més, s'ha afegit control d'errors complet.

Per altra banda, s'ha millorat substancialment la càrrega a memòria separant l'arxiu d'input en dos, un que conté els \textit{strings} de les paraules, i un altre que conté les normes i els \textit{embeddings} ja en binari, per estalviar la conversió a float en temps d'execució, a més que el fitxer és de menor mida en binari.

Finalment, s'ha afegit l'opció d'usar o no memòria \textit{pinned} segons una \textit{flag} de compilació.

\section*{Resultats}

Els resultats han sigut obtinguts en una màquina amb una GTX 2060 SUPER com a targeta gràfica, utilitzant un slot x16 PCIe 3.0, Intel I5-9600K com a CPU, i els models guardats en una NVME, Samsung 970 EVO, amb suficient RAM DDR4 com per a no observar thrashing. El sistema no te límits a pinned memory, per tant totes les execucions s'han fet utilitzant pinned memory.

En la implementació sequencial, com a mitja de 10 execucions, el temps d'execució mitjà d'una cerca es de 2425.9 ms, amb un error estàndard de 7.91ms, això significa que ha calculat 2.196.016 distàncies en una mitja de 905.2 distàncies calculades per ms.

En la última implementació de CUDA, com a mitja de 10 execucions, el temps d'execució mitjà d'una cerca es de 36.8ms, amb un error estàndard de 1.58ms, això significa que ha calculat 2.196.016 distàncies en una mitja de 59674.3 distàncies calculades per ms. De els 3 kernels, DotProduct, FirstMerge i BotchedMergeSort, DotProduct ocupa uns 22ms, First Merge 0.8ms, i BotchedMergeSort 1.3ms, la resta sent overhead, per tant FirstMerge i BotchedMergeSort son negligibles comparat amb DotProduct.


\begin{table}[h]
\centering
\begin{tabular}{|l|l|l|}
\hline
      & CPU  & GPU \\ \hline
ring  & 2405ms & 35ms  \\ \hline
key   & 2397ms & 46ms  \\ \hline
key   & 2410ms & 35ms  \\ \hline
key   & 2403ms & 46ms  \\ \hline
ring  & 2436ms & 31ms  \\ \hline
king  & 2468ms & 35ms  \\ \hline
king  & 2454ms & 35ms  \\ \hline
ring  & 2419ms & 35ms  \\ \hline
barca & 2414ms & 35ms  \\ \hline
messi & 2453ms & 35ms  \\ \hline
\end{tabular}
\end{table}




\begin{table}[h]
\centering
\begin{tabular}{|l|l|l|}
\hline
      & Speedup respecte anterior  & Speedup respecte CPU \\ \hline
Primera implementació  &  & ?? \\ \hline
Segona implementació   & ?? & 65.92  \\ \hline
\end{tabular}
\end{table}

\section*{Possibles millores}

\newpage
\bibliographystyle{ieeetr}
\bibliography{biblio}
\end{document}
